\documentclass[12pt,]{article}
\usepackage{lmodern}
\usepackage{amssymb,amsmath}
\usepackage{ifxetex,ifluatex}
\usepackage{fixltx2e} % provides \textsubscript
\ifnum 0\ifxetex 1\fi\ifluatex 1\fi=0 % if pdftex
  \usepackage[T1]{fontenc}
  \usepackage[utf8]{inputenc}
\else % if luatex or xelatex
  \ifxetex
    \usepackage{mathspec}
  \else
    \usepackage{fontspec}
  \fi
  \defaultfontfeatures{Ligatures=TeX,Scale=MatchLowercase}
\fi
% use upquote if available, for straight quotes in verbatim environments
\IfFileExists{upquote.sty}{\usepackage{upquote}}{}
% use microtype if available
\IfFileExists{microtype.sty}{%
\usepackage{microtype}
\UseMicrotypeSet[protrusion]{basicmath} % disable protrusion for tt fonts
}{}
\usepackage[margin=0.9in]{geometry}
\usepackage[unicode=true]{hyperref}
\PassOptionsToPackage{usenames,dvipsnames}{color} % color is loaded by hyperref
\hypersetup{
            pdftitle={Game Design Pitch},
            colorlinks=true,
            linkcolor=Maroon,
            citecolor=blue,
            urlcolor=blue,
            breaklinks=true}
\urlstyle{same}  % don't use monospace font for urls
\usepackage{natbib}
\bibliographystyle{plainnat}
\IfFileExists{parskip.sty}{%
\usepackage{parskip}
}{% else
\setlength{\parindent}{0pt}
\setlength{\parskip}{6pt plus 2pt minus 1pt}
}
\setlength{\emergencystretch}{3em}  % prevent overfull lines
\providecommand{\tightlist}{%
  \setlength{\itemsep}{0pt}\setlength{\parskip}{0pt}}
\setcounter{secnumdepth}{0}
% Redefines (sub)paragraphs to behave more like sections
\ifx\paragraph\undefined\else
\let\oldparagraph\paragraph
\renewcommand{\paragraph}[1]{\oldparagraph{#1}\mbox{}}
\fi
\ifx\subparagraph\undefined\else
\let\oldsubparagraph\subparagraph
\renewcommand{\subparagraph}[1]{\oldsubparagraph{#1}\mbox{}}
\fi

% set default figure placement to htbp
\makeatletter
\def\fps@figure{htbp}
\makeatother



% Stuff I added.
% --------------

\usepackage{indentfirst}
\usepackage[doublespacing]{setspace}
\usepackage{fancyhdr}
\pagestyle{fancy}
\usepackage{layout}   
\lhead{\sc Galloping Wheelbarrows}
\chead{}
\rhead{\thepage}
\lfoot{}
\cfoot{}
\rfoot{}

\renewcommand{\headrulewidth}{0.0pt}
\renewcommand{\footrulewidth}{0.0pt}

\usepackage{sectsty}
\sectionfont{\centering}
\subsectionfont{\centering}

\newtheorem{hypothesis}{Hypothesis}

% Begin document
% --------------

\begin{document}

\doublespacing

\begin{titlepage}
    \begin{center}
    \line(1,0){300} \\ 
    [0.25in]
    \huge{\bfseries Game Design Pitch} \\
    [2mm]
    \line(1,0){200} \\
    [1.5cm] 
    \textsc{\Large Galloping Wheelbarrows} \\
    [0.75cm]
    \textsc{\Large Can you deliver the dirt on time?} \\
    [9cm]
    \end{center}
    
    \begin{flushright}
    \textsc{\Large{Gurpreet Singh \\}\normalsize\emph{\ January 21, 2018 \\}\normalsize\emph{CS4483 \#250674134 \\} }
    
    \end{flushright}
    
\end{titlepage}

\newpage

\hypertarget{premise}{%
\subsection{Premise}\label{premise}}

The game will be an addictive 2D speed-based platform game and will
focus on fast moving action and racing. The characters is the game are
different types of wheelbarrows and the story revolves around delivering
precious dirt to the destination on time. The game will require fast
reactions and button spamming to win races. Single and multi-player game
modes will be available.

\hypertarget{player-motivation}{%
\subsection{Player Motivation}\label{player-motivation}}

The player will be driven to achieve the fastest record time on a map
and beat his peers in multi-player mode by mastering new techniques and
reacting faster to the changing map conditions. As the player progresses
they will unlock new maps but the basic movement techniques will remain
the same. The player just becomes better at incorporating them into
their play style to move faster.

The story mode of the game provides a timed challenge for the user to
complete and increases the number of different maps a player can play.
The story is also meant to be comedy so the player will have a good time
discovering what will happen next.

\hypertarget{genre}{%
\subsection{Genre}\label{genre}}

The genres for the game are speed-based, action, platformer and will be
played on the PC platform. Galloping Wheelbarrows will be following most
conventional game mechanics for these genres. One aspect that isn't seen
in other platformers is that it will be impossible to miss a platform,
and instead of having to aim for a platform, the user will have to
adjust their direction of travel to match where the next platform is
going to maintain momentum.

For example, one platform could be leading the player in the right
direction but then theres a loop into a jump and the next platform leads
left. If the user doesn't change their direction from right to left,
they will loss all their speed upon landing on the next platform

\hypertarget{gameplay}{%
\subsection{Gameplay}\label{gameplay}}

\hypertarget{gamemodes}{%
\subsubsection{Gamemodes}\label{gamemodes}}

In single player mode the player will try to get to the end of the map
in the least amount of time and in multi-player mode the player can race
against other players. The game will be played on a map per map basis
with a menu screen in between to switch between maps and progress in the
story.

\hypertarget{maps}{%
\subsubsection{Maps}\label{maps}}

A map will consist of a series of platforms going from the start of the
race to the end of a race. Some platforms may be reused but there will
always be one clear direction for the user to travel. Each platform will
direct the player in either the left or right direction. Each platform
can also have special qualities like speed increasing platforms,
inclined/declined platforms, and platforms that don't allow tricks on
them.

\hypertarget{player-controls}{%
\subsubsection{Player Controls}\label{player-controls}}

The player will use the left and right arrow keys to pick a direction to
move the player. Other controls include U to jump, E to flip right, O to
flip left, and A to do a spin. These controls will be re-bindable in the
settings menu and new controls may be added later in development to
include a wider range of tricks. When the player holds either left or
right the character slowly increases in speed the longer the key is
being held and the visual model of a wheelbarrow will show this by
increasingly leaning forward the longer the key is held.

\hypertarget{playing-a-round}{%
\subsubsection{Playing a Round}\label{playing-a-round}}

In each round of the game the user (a wheelbarrow) will be placed in the
center of the screen and the round will begin with a countdown. When the
countdown ends the user will be allowed the move the character and begin
racing from platform to platform and doing tricks to gain speed.

Each time the character successfully lands a trick, it gains a boost in
momentum. Chaining multiple tricks together without messing up will
allow for the fastest traversal through the map. In order to achieve
this the user will have to know or predict the map correctly because
tricks can be disrupted by certain platforms or increase/decreases in
elevation.

There is a stopwatch in the top right corner of the screen showing the
player how long they have taken to complete the map and if there are
multiple laps or not.

\hypertarget{movement-tricks}{%
\subsubsection{Movement Tricks}\label{movement-tricks}}

The player can use the flip keys or spin key to do a trick while they
are in the air. In order to gain a momentum boost from the trick, the
character has to land at the correct angle. For example if the character
is moving right, and does a front flip, the wheelbarrow has to be
leaning rightward to gain a momentum boost. If the wheelbarrow is
leaning backward the character will bounce and lose speed.

\hypertarget{competitive-analysis}{%
\subsection{Competitive Analysis}\label{competitive-analysis}}

Other games that operate with similar mechanics are Sonic the Hedgehog,
Freedom Planet and Uniracers. Uniracers is a direct competitor but is an
old game that has no chance of recreation. Freedom Planet and Sonic the
Hedgehog are not the same concept but target similar markets so they can
also be considered competition.

Sonic the Hedgehog popularized this play style but used it in more of
roleplaying fashion instead of racing. Freedom Planet expanded and
improved upon some of Sonic's best gameplay aspects but it remained
fundamentally the same in terms of gameplay. These games didn't utilize
the concept of doing tricks to gain more speed in the same way that
Galloping Wheelbarrows will and therefore they are fundamentally
different.

A major pro for these games is their's impressive map design and level
music. Those qualities will be very difficult to beat. A major con is
the lack of advanced movement mechanics in both games. Galloping
Wheelbarrows will be adding lots of complexity to movement and therefore
increase the skill ceiling.

Uniracers is a SNES game from 1994 that has similar movement mechanics
as Galloping Wheelbarrows but encountered a lawsuit from Pixar due to
graphics and therefore is no longer available to the mass audience.
Although, the general concept is similar, it has a very limited set of
tricks and characters that I would like to expand upon with Galloping
Wheelbarrows. It also does not have a story.

\hypertarget{unique-selling-proposition-usp}{%
\subsection{Unique Selling Proposition
(USP)}\label{unique-selling-proposition-usp}}

The concept of a fast paced wheelbarrow racing game has not been created
yet and I think there is a lot of potential to use this unusual
combination to appeal to gamers. The simple and unrestrictive game
mechanics of Galloping Wheelbarrows allows users to endlessly master and
exploit the maps to traverse them at insanely fast speeds.

This game will stand out from other retro games due to it's high skill
ceiling caused by advanced movement mechanics and funny story to ease
the player into the gameplay.

\hypertarget{story-synopsis}{%
\subsection{Story Synopsis}\label{story-synopsis}}

Galloping Wheelbarrows is not heavily based around the story therefore
it takes this opportunity to have a funny narrative before a map round
starts. The narrator takes the player through a story of how a boy needs
to deliver a wheelbarrow full of dirt to his father before nightfall.
That boy has a magic wheelbarrow that can do tricks and go very fast.
Then a game round starts with a wheelbarrow full of dirt and instead of
the usual stopwatch, a timer is started and the user must deliver the
dirt before the timer goes to zero. The main character is the
wheelbarrow and his name is Martin.

\hypertarget{target-market}{%
\subsection{Target Market}\label{target-market}}

The target market for this game would be PC gamers who enjoy retro style
games. People who have played uniracers in their childhood would really
enjoy this game as a modern evolution of that concept with good
graphics. The target demographic would be people old enough to have
played Uniracers on SNES but it wouldn't be restricted to that because
the game would be fun even if you didn't recognize it's predecessor
game. The game would aim to be ESRB rated E for Everyone.

\hypertarget{target-platform}{%
\subsection{Target Platform}\label{target-platform}}

Galloping Wheelbarrows will be created primarily for the GNU/Linux
platform and if possible later expanded to Windows. It will aim to
require very little hardware requirements and use minimal dependencies
to allow the most people to play it without difficulty.

\hypertarget{summary}{%
\subsection{Summary}\label{summary}}

In conclusion Galloping Wheelbarrows will be a game that makes many
people happy and endlessly challenge their quick thinking skills.

\end{document}

